\section{Simulation Study}
\label{simulation}

As the ensemble data described in \cref{case} cannot be published we instead provide a simulated exampel, with an identical framework.

\subsection{The Data}

The simulation data uses an identical set to the case data, 4 zones with different production capacity. The observational data is reused and thus identical to the case.

To replace the ensambles we simulated en ornstein-uhlenbeck (OU) mean-reverting process

\begin{equation}
    dX = \kappa\big(\mu(t) - X\big)\,dt + \sigma\, dB_t, 
    \qquad j = 1,\dots,n_{\mathrm{ensembles}}.
\end{equation}

The process was driven using the actual observations at each timepoint, scaled to the interval $\mu(t) = \frac{y_t}{c_{zone}}, \, c_{zone = \max y_t} $ Scaling was done simply to streamline the simulation code and allow parameters to be chosen independently of zones.

Simulation was performed using a euler-maruyama scheme, with timesteps of $dt = 1$. afterwards simulated ensembles were scaled back to original scale. 
\begin{equation}
    X_t = X_{t-1} + \kappa(\mu_{t} - X_{t-1}) + \sigma\epsilon_t
\end{equation}

where $\epsilon_t \sim N(0,1)$ and IID.

For this simulation study the parameters $\kappa = 0.33$ and $\sigma = 0.033$ was chosen.


\subsection{Procedure}

The procedure used to analise the simulation example is identical to the one used in \cref{case} for detail refer to this chapter.

\begin{figure}[t]
\centering

% ---------- Style definitions ----------
\tikzset{
  >=Latex,
  box/.style={draw, rounded corners, thick, align=center, inner sep=4pt, minimum width=38mm, minimum height=9mm, fill=orange!10},
  seq/.style={draw, rounded corners, thick, align=center, inner sep=4pt, minimum width=44mm, minimum height=10mm, fill=purple!10},
  head/.style={draw, rounded corners, thick, align=center, inner sep=4pt, minimum width=38mm, minimum height=9mm, fill=blue!8},
  outnode/.style={draw, circle, thick, minimum size=4mm, inner sep=1pt, fill=green!10},
  note/.style={font=\small, align=center}
}

% ---------- Simple model ----------
\begin{minipage}{0.47\linewidth}
\centering
\textbf{Simple model}

\begin{tikzpicture}[node distance=6mm and 10mm]
  % Nodes
  \node[box] (inpS) {Sequence input $(T \times d_x)$};
  \node[head, below=of inpS] (denseS) {Dense ($1$) \\ Activation: Identity};

  % Quantile head (three example outputs + ellipsis)
  \node[outnode, below left=7mm and -8mm of denseS] (q1) {$q_{\alpha_1}$};
  \node[below=7mm of denseS] (qdots) {$\cdots$};
  \node[outnode, below right=7mm and -8mm of denseS] (qm) {$q_{\alpha_m}$};
  \node[note, below=3mm of qdots] (noteS) {Outputs: $m$ quantiles $\{\alpha_i\}$};

  % Arrows
  \draw[->, thick] (inpS) -- (denseS);
  \draw[->, thick] (denseS) -- (q1);
  \draw[->, thick] (denseS) -- (qm);
\end{tikzpicture}
\end{minipage}
\hfill
% ---------- Latent LSTM model ----------
\begin{minipage}{0.47\linewidth}
\centering
\textbf{Latent LSTM model}

\begin{tikzpicture}[node distance=6mm and 10mm]
  % Nodes
  \node[box] (inpL) {Sequence input $(d_t \time d_x)$};
  \node[seq, below=of inpL] (proj1) {Dense \\ Width: $h$ \quad Act: SELU};
  \node[seq, below=of proj1] (proj2) {Dense \\ Width: $h$ \quad Act: SELU};
  \node[seq, below=of proj2] (lstm) {LSTM \quad Hidden: $h$};
  \node[head, below=of lstm] (headL) {Dense ($h$) \\ Activation: SELU};
  \node[head, below=of headL] (outL) {Dense ($h$) \\ Activation: Identity};


  % Quantile head (three example outputs + ellipsis)
  \node[outnode, below left=7mm and -8mm of outL] (q1L) {$q_{\alpha_1}$};
  \node[below=7mm of outL] (qdotsL) {$\cdots$};
  \node[outnode, below right=7mm and -8mm of outL] (qmL) {$q_{\alpha_m}$};
  \node[note, below=3mm of qdotsL] (noteS) {Outputs: $m$ quantiles $\{\alpha_i\}$};

  % Arrows
  \draw[->, thick] (inpL) -- (proj1);
  \draw[->, thick] (proj1) -- (proj2);
  \draw[->, thick] (proj2) -- (lstm);
  \draw[->, thick] (lstm) -- (headL);
  \draw[->, thick] (headL) -- (outL);
  \draw[->, thick] (outL) -- (q1L);
  \draw[->, thick] (outL) -- (qdotsL);
  \draw[->, thick] (outL) -- (qmL);
\end{tikzpicture}
\end{minipage}

\caption{Illustrations of the two neural network architectures. Left: a Simple model with a linear (Identity) output head producing $m$ quantile levels $\{\alpha_i\}$. Right: a Latent model with two SELU-activated projections applied per time step, followed by an LSTM encoder, a SELU head, and an Identity output to $m$ quantiles. Symbols: $T$ sequence length, $d_x$ input feature dimension, $d_\ell$ latent width, $h$ LSTM hidden size, $m$ number of quantiles.}
\label{fig:nn_architectures}
\end{figure}


The final scores of running the method using simulated ensembles can be seen in \cref{tab_scores_simulated}


\begin{table}[htb]
\centering
\caption[Ensemble summary]{Model Scores (Ensemble in Original Units; Other Models as Percent of Ensemble)}
\label{tab:ensemble_combined}
\begin{tabular}{llrrrr}
\toprule
 &  & MAE & RMSE & CRPS & VarS \\
\midrule
\multirow[c]{5}{*}{DK1-onshore} & Ensembles & 325.60 & 439.79 & 236.15 & 68.71 \\
\cline{2-6}
 & Simple - Marginal & \bfseries 36.7\% & \bfseries 36.7\% & \bfseries 40.6\% & \bfseries 78.6\% \\
\cline{2-6}
 & Latent - Marginal & \bfseries 32.0\% & \bfseries 32.6\% & \bfseries 33.1\% & \bfseries 45.1\% \\
\cline{2-6}
 & Simple & \bfseries 36.6\% & \bfseries 36.7\% & \bfseries 37.8\% & \bfseries 43.9\% \\
\cline{2-6}
 & Latent & \bfseries 32.0\% & \bfseries 32.5\% & \bfseries 32.5\% & \bfseries 35.0\% \\
\cline{1-6} \cline{2-6}
\multirow[c]{5}{*}{DK2-onshore} & Ensembles & 62.57 & 83.80 & 44.45 & 13.29 \\
\cline{2-6}
 & Simple - Marginal & \bfseries 37.3\% & \bfseries 37.7\% & \bfseries 41.0\% & \bfseries 73.5\% \\
\cline{2-6}
 & Latent - Marginal & \bfseries 37.4\% & \bfseries 37.5\% & \bfseries 39.3\% & \bfseries 52.2\% \\
\cline{2-6}
 & Simple & \bfseries 37.4\% & \bfseries 37.8\% & \bfseries 40.6\% & \bfseries 61.9\% \\
\cline{2-6}
 & Latent & \bfseries 37.4\% & \bfseries 37.6\% & \bfseries 39.7\% & \bfseries 46.6\% \\
\cline{1-6} \cline{2-6}
\multirow[c]{5}{*}{DK1-offshore} & Ensembles & 158.79 & 206.86 & 110.34 & 35.17 \\
\cline{2-6}
 & Simple - Marginal & \bfseries 43.6\% & \bfseries 45.3\% & \bfseries 48.4\% & \bfseries 72.0\% \\
\cline{2-6}
 & Latent - Marginal & \bfseries 42.9\% & \bfseries 44.4\% & \bfseries 45.5\% & \bfseries 61.1\% \\
\cline{2-6}
 & Simple & \bfseries 43.6\% & \bfseries 45.3\% & \bfseries 47.2\% & \bfseries 55.3\% \\
\cline{2-6}
 & Latent & \bfseries 42.9\% & \bfseries 44.4\% & \bfseries 45.1\% & \bfseries 50.5\% \\
\cline{1-6} \cline{2-6}
\multirow[c]{5}{*}{DK2-offshore} & Ensembles & 131.31 & 173.00 & 91.49 & 32.27 \\
\cline{2-6}
 & Simple - Marginal & \bfseries 42.0\% & \bfseries 44.6\% & \bfseries 45.8\% & \bfseries 60.8\% \\
\cline{2-6}
 & Latent - Marginal & \bfseries 39.5\% & \bfseries 41.9\% & \bfseries 41.6\% & \bfseries 49.3\% \\
\cline{2-6}
 & Simple & \bfseries 42.1\% & \bfseries 44.6\% & \bfseries 45.4\% & \bfseries 48.9\% \\
\cline{2-6}
 & Latent & \bfseries 39.5\% & \bfseries 41.9\% & \bfseries 41.6\% & \bfseries 43.5\% \\
\cline{1-6} \cline{2-6}
\bottomrule
\end{tabular}
\end{table}
